To answer the main questions specified in the testplan, we found that the amount of information contained on the product pages, caused the participants to be confused when searching for specific information, such as information regarding shipping. Another problem we noticed during the evaluation, was that product pages were structured in such a way that details about the given product was spread very widely across the page, which caused the participants to become frustrated.

As previously mentioned, we observed that they did not utilize the recommender system, among other functionality, even though we attempted to create assignments that facilitated exploration of Amazon. The primary reason for this, is that they found it more convenient to use the main search bar for navigation of the entire system. 

\subsection{For Co-discovery}
We found that using Co-discovery facilitated in more exploration of the system, as the participants naturally influenced each other during their use of the system. This in turn also made them less likely to interact with the moderator. Co-discovery also made it more natural to think aloud for the participants, in the sense that it took place in the form of an ongoing conversation. This interaction between the participants helped highlight their false assumptions regarding the system. 

However, the method has some cons as well. Due to the participants influencing each other, the method does not tell anything about how the individual would have interacted with the system. Another problem is that being able to perform the evaluation requires more resources, compared to methods requiring only one participant, partly because solving the logistics of scheduling participants, capable of cooperating, can be difficult and time consuming.

Some of the issues we encountered were related to the assignments we proposed. An example of such an issue is that the participants did not explore much of the product page. A cause for this may be that the assignments did not provide significantly motivation for it. A way to improve this, could be to make participants look up further product information forcing them to explore product pages in their entirety. 

We also experienced that some of our assignments were not formulated restrictively enough, causing the participants to explore less of the system.
\bjarke{Co-discovery works best for discovering pragmatic issues with the system}

\subsection{For AttrakDiff}
We experienced that the AttrakDiff questionare was quick and simple to create, and that it was easy to get an overview of the opinions of all the participants. However, we found that the participants had a hard time interpreting the meaning of the questions. Another problem is the sample size, which was simply too small to make any conclusions. This suggests that AttrakDiff should be used in contexts that ensure a large number of participants, e.g. questionares by email. The questions are also not very informative, which makes it hard to make conclusions on why the participant thinks what he thinks, in turn making it hard to utilize the results. It can also occur that the participants don't put much effort into filling in the questions, rendering the results useless.
\bjarke{did the translation to Danish help...?}

\subsection{Advice for Amazon}
To summarize, the overall problems identified in regards to UX on Amazon are:
\begin{itemize}
	\item Too much of information and hidden information.
	\item A lot of diversity and inconsistency in product pages in regards to information and functionality.
	\item A lot of functionality serving the same purpose, while quite a few of the features are not used. 
\end{itemize}

We propose the following advice for Amazon:

\begin{enumerate}
\item Make important hidden information more visible and accessible to the users.
\item Investigate whether it is possible to cut down on redundant information and functionality on product pages.
\item Restructure product pages to make them more consistent.
\end{enumerate}

\subsection{Preferred method}
Provided we were to perform another UX evaluation, under the same circumstances, we would choose to use the method Co-discovery. A reason for this is that it naturally facilitates interaction between the participants, which results in further exploration of the system. While AttrakDiff in this setting is rendered somewhat ineffective, due to the requirement of a large sample size.